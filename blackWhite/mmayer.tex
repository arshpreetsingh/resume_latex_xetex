%%%%%%%%%%%%%%%%%
% This is an example CV created using altacv.cls (v1.1.5, 1 December 2018) written by
% LianTze Lim (liantze@gmail.com), based on the
% Cv created by BusinessInsider at http://www.businessinsider.my/a-sample-resume-for-marissa-mayer-2016-7/?r=US&IR=T
%
%% It may be distributed and/or modified under the
%% conditions of the LaTeX Project Public License, either version 1.3
%% of this license or (at your option) any later version.
%% The latest version of this license is in
%%    http://www.latex-project.org/lppl.txt
%% and version 1.3 or later is part of all distributions of LaTeX
%% version 2003/12/01 or later.
%%%%%%%%%%%%%%%%

%% If you are using \orcid or academicons
%% icons, make sure you have the academicons
%% option here, and compile with XeLaTeX
%% or LuaLaTeX.
% \documentclass[10pt,a4paper,academicons]{altacv}

%% Use the "normalphoto" option if you want a normal photo instead of cropped to a circle
% \documentclass[10pt,a4paper,normalphoto]{altacv}

\documentclass[10pt,a4paper,ragged2e]{altacv}

%% AltaCV uses the fontawesome and academicon fonts
%% and packages.
%% See texdoc.net/pkg/fontawecome and http://texdoc.net/pkg/academicons for full list of symbols. You MUST compile with XeLaTeX or LuaLaTeX if you want to use academicons.

% Change the page layout if you need to
\geometry{left=2cm,right=10cm,marginparwidth=6.8cm,marginparsep=1.2cm,top=1.25cm,bottom=1.25cm}

% Change the font if you want to, depending on whether
% you're using pdflatex or xelatex/lualatex
\ifxetexorluatex
  % If using xelatex or lualatex:
  \setmainfont{Carlito}
\else
  % If using pdflatex:
  \usepackage[utf8]{inputenc}
  \usepackage[T1]{fontenc}
  \usepackage[default]{lato}
\fi

% Change the colours if you want to
\definecolor{VividPurple}{HTML}{000000}
\definecolor{SlateGrey}{HTML}{2E2E2E}
\definecolor{LightGrey}{HTML}{2E2E2E}
\colorlet{heading}{VividPurple}
\colorlet{accent}{VividPurple}
\colorlet{emphasis}{SlateGrey}
\colorlet{body}{LightGrey}

% Change the bullets for itemize and rating marker
% for \cvskill if you want to
\renewcommand{\itemmarker}{{\small\textbullet}}
\renewcommand{\ratingmarker}{\faCircle}

%% sample.bib contains your publications
\addbibresource{sample.bib}

\begin{document}
\name{Arshpreet Singh}
\tagline{Computer Science Engineer, B.Tech (2018-2022)}
% Cropped to square from https://en.wikipedia.org/wiki/Marissa_Mayer#/media/File:Marissa_Mayer_May_2014_(cropped).jpg, CC-BY 2.0
%\photo{3.3cm}{profile.jpg}
\personalinfo{%
  % Not all of these are required!
  % You can add your own with \printinfo{symbol}{detail}
  \email{pnishant46@gmail.com}
%   \phone{000-00-0000}
%  \mailaddress{Address, Street, 00000 County}
  \location{Tirap, Arunachal Pradesh,India}
%  \homepage{marissamayr.tumblr.com/}
%  \twitter{@marissamayer}
   \github{github.com/nipan09} % I'm just making this up though.
%   \orcid{orcid.org/0000-0000-0000-0000} % Obviously making this up too. If you want to use this field (and also other academicons symbols), add "academicons" option to \documentclass{altacv}
}

%% Make the header extend all the way to the right, if you want.
\begin{fullwidth}
\makecvheader
\end{fullwidth}

%% Depending on your tastes, you may want to make fonts of itemize environments slightly smaller
\AtBeginEnvironment{itemize}{\small}

%% Provide the file name containing the sidebar contents as an optional parameter to \cvsection.
%% You can always just use \marginpar{...} if you do
%% not need to align the top of the contents to any
%% \cvsection title in the "main" bar.
\cvsection[page1sidebar]{Experience}

\cvevent{Contributor In FOSS@Amrita}{Amrita Vishwa Vidyapeetham, Amritapuri}{July 2019 -- Present}{Kollam, Kerala, India}
\begin{itemize}
\item FOSS@Amrita is a Free and Open Source Software club of Amrita Vishwa Vidyapeetham. I am currently contributing in a django based question paper generator web app 'Qujini'.

\end{itemize}

\divider

\cvevent{Participated In Techradian Python Workshop}{Amrita Vishwa Vidyapeetham, Amritapuri}{Mar 2019 -- Apr 2018}{Kollam, Kerala, India}
\begin{itemize}
\item It was a three days python workshop cum contest held in association with TechFest, IIT Bombay and I was ranked among top 10 candidates in the contest.

\end{itemize}

\divider

\cvevent{Participated In IEEE Machine Learning Workshop}{Amrita Vishwa Vidyapeetham, Amritapuri}{Jan 2019 -- Jan 2019}{Kollam, Kerala,India}
\begin{itemize}
\item It was a brief introduction on ML and its branches conducted by IEEE students club of Amrita Vishwa Vidyapeetham.
\end{itemize}
 
\divider

\cvevent{A Member Of Code@Amrita}{Amrita Vidya Vidaypeetham, Amritapuri}{Sep 2018 -- Dec 2018}{Kollam, Kerala, India}
\begin{itemize}
\item Code@Amrita is a student club for competitive programmers. Being an active member in the club, I participated in the coding campaign conducted by the club in Banglore.
\end{itemize}

%\divider

\cvsection{Achievements}
\smallskip
\begin{itemize}
\item Ranked among top 10 finalist in Tehradian python contest.
\smallskip
\item Secured 8.8/10 CGPA in my first year of graduation.
\smallskip
\item Secured AIR 25764 in JEE Mains 2018
\smallskip
\item Secured 83.4 percentage in Higher Secondary School Examination.
\smallskip
\item Secured 9.4/10 CGPA in Secondary School Examination.
\end{itemize}

\cvsection{SKILLS}

\cvskill{C, C++}{3}
\cvskill{Python}{4}
\cvskill{Django, HTML}{3}
\cvskill{Bash Scripting}{3}
\cvskill{Linux (Ubuntu)}{4}
%\divider

% \cvevent{Product Engineer}{Google}{23 June 1999 -- 2001}{Palo Alto, CA}

% \begin{itemize}
% \item Joined the company as employe \#20 and female employee \#1
% \item Developed targeted advertisement in order to use user's search queries and show them related ads
% \end{itemize}

%\cvsection{A Day of My Life}

% Adapted from @Jake's answer from http://tex.stackexchange.com/a/82729/226
% \wheelchart{outer radius}{inner radius}{
% comma-separated list of value/text width/color/detail}
% Some ad-hoc tweaking to adjust the labels so that they don't overlap
% \wheelchart{1.5cm}{0.5cm}{%
%   10/10em/accent!30/Sleeping \& dreaming about work,
%   25/9em/accent!60/Public resolving issues with Yahoo!\ investors,
%   5/13em/accent!10/\footnotesize\\[1ex]New York \& San Francisco Ballet Jawbone board member,
%   20/15em/accent!40/Spending time with family,
%   5/8em/accent!20/\footnotesize Business development for Yahoo!\ after the Verizon acquisition,
%   30/9em/accent/Showing Yahoo!\ employees that their work has meaning,
%   5/8em/accent!20/Baking cupcakes
% }

\clearpage

% \cvsection[page2sidebar]{Publications}

\nocite{*}

% \printbibliography[heading=pubtype,title={\printinfo{\faBook}{Books}},type=book]

% \divider

% \printbibliography[heading=pubtype,title={\printinfo{\faFileTextO}{Journal Articles}}, type=article]

% \divider

% \printbibliography[heading=pubtype,title={\printinfo{\faGroup}{Conference Proceedings}},type=inproceedings]

% %% If the NEXT page doesn't start with a \cvsection but you'd
% %% still like to add a sidebar, then use this command on THIS
% %% page to add it. The optional argument lets you pull up the
% %% sidebar a bit so that it looks aligned with the top of the
% %% main column.
% % \addnextpagesidebar[-1ex]{page3sidebar}


\end{document}

